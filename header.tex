\documentclass%
[aspectratio=1610]%
{beamer}

\usepackage{hyperref}

\usepackage[
    locale=DE,
    separate-uncertainty=true,  % use \pm
    per-mode=symbol-or-fraction,
    %per-mode=reciprocal,
    %output-decimal-marker=.,
]{siunitx}

\usepackage{subfig} 
%The pack­age pro­vides sup­port for the ma­nip­u­la­tion and ref­er­ence 
%of small or ‘sub’ fig­ures and ta­bles within a sin­gle fig­ure or ta­ble en­vi­ron­ment. 
%It is con­ve­nient to use this pack­age when your sub­fig­ures are to be sep­a­rately cap­tioned, 
%ref­er­enced, or are to be in­cluded in the List-of-Fig­ures. A new \sub­fig­ure com­mand 
%is in­tro­duced which can be used in­side a fig­ure en­vi­ron­ment for each sub­fig­ure. 
%An op­tional first ar­gu­ment is used as the cap­tion for that sub­fig­ure. 

\usepackage{lmodern}
%freie Skallierbarkeit der Schriftgröße mit {\fontsize{Fontgröße}{Grundlinienabstand} \selectfont}

\usepackage{amssymb}
%Das Paket amssymb stellt einen erweiterten Weg dar alle Zeichen aus msam und msmb nutzbar zu machen.

\usepackage{mathtools}
%lädt amsmath selber, \usefonttheme{professionalfonts}

\usepackage{fontspec}
%Fontspec is a pack­age for XeLaTeX and LuaLaTeX. It pro­vides an au­to­matic and uni­fied in­ter­face 
%to fea­ture-rich AAT and OpenType fonts through the NFSS in LaTeX run­ning on XeTeX or LuaTeX en­gines
\usepackage{polyglossia}
\setmainlanguage{german}
\usepackage{csquotes}

\usepackage{tikz}

\usepackage{xcolor}

% Attempts to make animation work
% \usepackage{movie15}
%\usepackage{animate}

\institute[Tutorium]

\usepackage[
    math-style=ISO,
    bold-style=ISO,
    nabla=upright,
    partial=upright,
    sans-style=italic,
]{unicode-math}

% \setmainfont{Fira Sans}
% \setmathfont{Fira Math}
% Nutzt Max Nöthe

\usetheme{tudo}
% Thanks Max

\title{Tutorium SS 19}
\author[M. Bourgett \& J.Herdieckerhoff]{Jan Herdieckerhoff und Mirjam Bourgett}
\date{April 2019}
