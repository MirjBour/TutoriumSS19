\begin{frame}[t]{Aufgabe 1 - Divergenz, Rotation und Gradient}

    Gegeben sei der allgemeine Ortsvektor $\vec{r}$ mit $r := |\vec{r}|$ und $r > 0$. Berechnen Sie: 
    
    (a) $\vec{\nabla}\, (x^2 + xz - z^2 +3xyz)$
    
(b) $\vec{\nabla} \cdot [20xz - 2x^2 + 8x,\, 2e^z -1 + y\, (sin^2(xyz) + \{e^{ixyz} + i^3 sin(xyz)\}^2), ln(y^7) + 46xz + 33z + 11z^2]^\text{T}$
\\    
    (c) $\vec{\nabla} \times (2y -4,\, 4z, x^2 + y^2 + z^2)^\text{T}$

    (d) $\vec{\nabla}\, r$
    
    (e) $\vec{\nabla} \times \vec{r}$
    
    (f) $\vec{\nabla} \cdot \vec{r}$
    
    (g) $\vec{\nabla} \frac{1}{r}$
    
    (h) $\vec{\nabla} \cdot (\frac{1}{r^2} \vec{e_\text{r}})$

\end{frame}


\begin{frame}[t]{Aufgabe 2 - Durchschlagfeldstärke von Luft}

    Bei der Durchschlagsfeldstärke von ca. $\SI{3e6}{\volt\per\meter}$ werden freie Elektronen in der Luft so stark
beschleunigt, dass sie umgebende Moleküle ionisieren. Die Luft wird dadurch leitfähig.
\\
(a) Wie viel Ladung kann auf einer Kugel mit einem Durchmesser von $\SI{20}{\centi\meter}$ maximal gesammelt
werden, bevor an ihrer Oberfläche die Durchschlagsfeldstärke erreicht wird?
\\
(b) Stellen Sie sich einen Gewitterblitz als gerade, beliebig dünne Linie mit konstanter linearer
Ladungsdichte (Ladung pro Länge) vor. In welchem Radius würde er die Luft ionisieren, wenn
seine Ladungsdichte $\lambda = \SI{1e-3}{\coulomb\per\meter} $ betragen würde?

\end{frame}

\begin{frame}[t]{Aufgabe 3 - Elektrische Ladung der Erde}

Nahe der Erdoberfläche kann ein elektrisches Feld nachgewiesen werden. Es ist vertikal nach unten
gerichtet und beträgt im Mittel $\SI{150}{\newton\per\coulomb}$ (mit starken zeitlichen und örtlichen Schwankungen).

(a) Berechnen Sie mithilfe des Gaußschen Gesetzes die durchschnittliche Flächenladungsdichte
an der Erdoberfläche, wenn man die Erde als leitende Kugel auffasst.

(b) Wie groß wäre demnach die Gesamtladung der Erde (mittlerer Erdradius $r_E = \SI{6371}{\kilo\meter}$)?

(c) Zwei Kugeln der Masse $\SI{100}{\gram}$ werden aus einer Höhe von $\SI{2}{\meter}$ fallengelassen. Eine ist elektrisch
neutral, die andere trägt eine Ladung von $+\SI{100}{\micro\coulomb}$. Welche Kugel fällt schneller und um wie
viel unterscheidet sich die Fallzeit? Vernachlässigen Sie den Luftwiderstand.

\end{frame}


\begin{frame}[t]{Aufgabe 4 - Gewitterwolke}

Eine Gewitterwolke mit $\SI{17}{\kilo\meter\squared}$ Gesamtfläche schwebt in \SI{900}{\meter} Höhe über der Erdoberfläche. Die
Wolke bildet zusammen mit der Erdoberfläche einen Plattenkondensator.\\

(a) Berechnen Sie die Kapazität dieses Plattenkondensators (die begrenzende Fläche auf der Erde
sei gleich der Wolkenfläche).\\

(b) Wie groß kann die Ladung der Gewitterwolke werden, bis sich der Kondensator über einen
Blitz entlädt? Die Durchschlagsfeldstärke von Luft beträgt \SI{3e6}{\volt\per\meter}.\\

(c) Der Kondensator wird, wenn er die kritische Spannung erreicht, durch einen Blitz vollständig
entladen. Welcher mittlere Strom in Ampere ($\SI{1}{\ampere} = \SI{1}{\coulomb\per\second}$) fließt zur Erde, wenn der Blitz
$\SI{1}{\milli\second}$ dauert?
\end{frame}
