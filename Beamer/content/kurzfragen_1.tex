\begin{frame}{Kurzfragen}
    \begin{itemize}
        \item Nenne 5 Eigenschaften von Feldlinien und ihr Bedeutung.
        \item <2-> Wie sieht der Feldlinienverlauf bei einer metallischen Hohlkugel mit Radius R aus in der sich eine\\
        Ladung Q bei R/2 befindet? 
        \item <3-> Wie sieht das Feld außen aus? Gibt es ein Feld außen?
        \item <4-> Was ist DIE Aussage vom Gaußschen Gesetz für Magnet felder also:
        \begin{align*}
            \vec{\nabla} \cdot \vec{B} = 0
        \end{align*}
        \item <4-> Was könnte man als analoges Gesetz zum Coulombgesetz der Elektrostatik in der Magnetostatik nennen?
    \end{itemize}
    
\end{frame}\begin{frame}{Kurzfragen}
    \begin{itemize}
        \item Erkläre für deine Eltern tauglich die Größen Gradient, Divergenz und Rotation. 
        \item <2-> Was sagt das Faradaysche Induktionsgesetz aus?
        \item <3-> Wie kann sich der magnetische Fluss ändern?
        \item <4-> Was ist die Ursache für den Verschiebungsstrom?       
    \end{itemize}    
\end{frame}
